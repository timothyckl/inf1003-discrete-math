\documentclass[12pt]{scrartcl}

\usepackage{amsmath,amssymb}
\usepackage{fullpage}
\usepackage{enumitem}
\usepackage{hyperref}

\setlength{\parindent}{0pt}

\begin{document}

\begin{center}
	\hrule
	\vspace{0.4cm}
	{\textbf{\large INF1003 Tutorial 2}}\\[0.2cm]
\end{center}

\textbf{Name:} Timothy Chia\hspace{\fill} \textbf{Topic:} Number Theory\\
\textbf{Student ID:} 2501530 \hspace{\fill} \textbf{Due Date:} 14th September 2025, 10:00 PM\\

\hrule

\begin{enumerate}[label=\textbf{\arabic*.}]

	%------------------------------------------------------------
	\item Evaluate the following:
	      \begin{enumerate}[label=(\alph*)]

		      \item $13 \bmod 3$.

		            \textbf{Solution.}
		            We divide $13$ by $3$:
		            \[
			            13 = 3 \cdot 4 + 1.
		            \]
		            The remainder is $1$, so
		            \[
			            13 \bmod 3 = 1.
		            \]

		            \medskip

		      \item $-22 \bmod 23$.

		            \textbf{Solution.}
		            We want the remainder in the range $0 \le r \le 22$.

		            Note that
		            \[
			            -22 + 23 = 1,
		            \]
		            so $-22$ and $1$ differ by a multiple of $23$:
		            \[
			            -22 \equiv 1 \pmod{23}.
		            \]
		            Therefore
		            \[
			            -22 \bmod 23 = 1.
		            \]

	      \end{enumerate}

	      \vspace{0.4cm}

	      %------------------------------------------------------------
	\item Find the integer $a$ such that
	      \[
		      a \equiv -15 \pmod{27}
		      \quad\text{and}\quad
		      0 \le a \le 26.
	      \]

	      \textbf{Solution.}
	      We add $27$ to $-15$ to obtain an equivalent non-negative representative:
	      \[
		      -15 + 27 = 12.
	      \]
	      Hence
	      \[
		      -15 \equiv 12 \pmod{27},
	      \]
	      so the required integer is
	      \[
		      a = 12.
	      \]

	      \vspace{0.4cm}

	      %------------------------------------------------------------
	\item Decide whether $80$ and/or $103$ are congruent to $5 \pmod{17}$.

	      \textbf{Solution.}
	      Compute each number modulo $17$.

	      For $80$:
	      \[
		      80 = 17 \cdot 4 + 12 \quad\Rightarrow\quad 80 \equiv 12 \pmod{17}.
	      \]

	      For $103$:
	      \[
		      103 = 17 \cdot 6 + 1 \quad\Rightarrow\quad 103 \equiv 1 \pmod{17}.
	      \]

	      We compare with $5$:
	      \[
		      12 \not\equiv 5 \pmod{17},\qquad
		      1 \not\equiv 5 \pmod{17}.
	      \]

	      Therefore, neither $80$ nor $103$ is congruent to $5 \pmod{17}$.

	      \vspace{0.4cm}

	      %------------------------------------------------------------
	\item Compute
	      \[
		      (-133 \bmod 23) + (26 \bmod 23).
	      \]

	      \textbf{Solution.}
	      First find each remainder separately.

	      For $-133 \bmod 23$:
	      \[
		      23 \cdot 6 = 138,\quad
		      -133 + 138 = 5,
	      \]
	      so
	      \[
		      -133 \equiv 5 \pmod{23}
		      \quad\Rightarrow\quad
		      -133 \bmod 23 = 5.
	      \]

	      For $26 \bmod 23$:
	      \[
		      26 = 23 \cdot 1 + 3
		      \quad\Rightarrow\quad
		      26 \bmod 23 = 3.
	      \]

	      Add the two remainders:
	      \[
		      (-133 \bmod 23) + (26 \bmod 23) = 5 + 3 = 8.
	      \]

	      So the value of the expression is
	      \[
		      8.
	      \]

	      \vspace{0.4cm}

	      %------------------------------------------------------------
  \newpage
	\item Compute
	      \[
		      (89^3 \bmod 74)^4 \bmod 26.
	      \]

	      \textbf{Solution.}
	      Step 1: Compute $89^3 \bmod 74$.

	      First reduce $89$ modulo $74$:
	      \[
		      89 = 74 + 15 \quad\Rightarrow\quad 89 \equiv 15 \pmod{74}.
	      \]
	      Then
	      \[
		      89^2 \equiv 15^2 = 225 \pmod{74}.
	      \]
	      Now reduce $225$ modulo $74$:
	      \[
		      74 \cdot 3 = 222,\quad 225 - 222 = 3,
	      \]
	      so
	      \[
		      89^2 \equiv 3 \pmod{74}.
	      \]
	      Next,
	      \[
		      89^3 \equiv 89^2 \cdot 89 \equiv 3 \cdot 15 = 45 \pmod{74}.
	      \]
	      Thus
	      \[
		      89^3 \bmod 74 = 45.
	      \]

	      Step 2: Compute $45^4 \bmod 26$.

	      First reduce $45$ modulo $26$:
	      \[
		      45 = 26 + 19 \quad\Rightarrow\quad 45 \equiv 19 \pmod{26}.
	      \]
	      So
	      \[
		      45^4 \equiv 19^4 \pmod{26}.
	      \]

	      Compute $19^2$ modulo $26$:
	      \[
		      19^2 = 361,\quad 26 \cdot 13 = 338,\quad 361 - 338 = 23,
	      \]
	      so
	      \[
		      19^2 \equiv 23 \pmod{26}.
	      \]

	      Then
	      \[
		      19^4 = (19^2)^2 \equiv 23^2 = 529 \pmod{26}.
	      \]
	      Now reduce $529$ modulo $26$:
	      \[
		      26 \cdot 20 = 520,\quad 529 - 520 = 9,
	      \]
	      so
	      \[
		      19^4 \equiv 9 \pmod{26}.
	      \]

	      Therefore
	      \[
		      (89^3 \bmod 74)^4 \bmod 26 = 9.
	      \]

	      \vspace{0.4cm}

	      %------------------------------------------------------------
  \newpage
	\item Determine whether each of these integers is prime.
	      \begin{enumerate}[label=(\alph*)]

		      \item $97$
		      \item $111$
		      \item $187$

	      \end{enumerate}

	      \textbf{Solution.}

	      \begin{enumerate}[label=(\alph*)]

		      \item $97$.

		            We check divisibility by primes up to $\sqrt{97} \approx 9.8$, i.e.\ $2,3,5,7$.

		            \begin{itemize}
			            \item $97$ is odd, so not divisible by $2$.
			            \item Sum of digits is $9+7=16$, not a multiple of $3$, so not divisible by $3$.
			            \item It does not end with $0$ or $5$, so not divisible by $5$.
			            \item $97 / 7 = 13$ remainder $6$, so not divisible by $7$.
		            \end{itemize}

		            Since there is no prime divisor $\leq \sqrt{97}$, $97$ is prime.

		            \medskip

		      \item $111$.

		            Sum of digits: $1 + 1 + 1 = 3$, which is divisible by $3$, hence
		            \[
			            111 \text{ is divisible by } 3.
		            \]
		            Indeed
		            \[
			            111 = 3 \cdot 37.
		            \]
		            Therefore $111$ is not prime.

		            \medskip

		      \item $187$.

		            We test small primes:

		            \begin{itemize}
			            \item $187$ is odd, so not divisible by $2$.
			            \item Sum of digits is $1+8+7 = 16$, not divisible by $3$.
			            \item It does not end with $0$ or $5$, so not divisible by $5$.
		            \end{itemize}

		            Try dividing by $11$:
		            \[
			            11 \cdot 17 = 187.
		            \]
		            So $187 = 11 \cdot 17$, and hence is not prime.

	      \end{enumerate}

	      \vspace{0.4cm}

	      %------------------------------------------------------------
	\item Find the prime factorisation of $1001$ and $1111$.

	      \textbf{Solution.}

	      For $1001$:
	      \[
		      1001 = 7 \cdot 143,\quad 143 = 11 \cdot 13.
	      \]
	      Therefore
	      \[
		      1001 = 7 \cdot 11 \cdot 13.
	      \]

	      For $1111$:

	      First note that $1111$ ends with $1$, so it is not divisible by $2$ or $5$.
	      Check divisibility by $11$:
	      \[
		      1111 / 11 = 101,
	      \]
	      so
	      \[
		      1111 = 11 \cdot 101.
	      \]
	      Since $101$ is prime, this is the prime factorisation.

	      \vspace{0.4cm}

	      %------------------------------------------------------------
	\item Determine whether the integers in each of these sets are pairwise relatively prime.
	      \begin{enumerate}[label=(\alph*)]

		      \item $11, 15, 19$
		      \item $12, 17, 31, 37$

	      \end{enumerate}

	      \textbf{Solution.}

	      A set of integers is \emph{pairwise relatively prime} if the gcd of every pair of distinct integers in the set is $1$.

	      \begin{enumerate}[label=(\alph*)]

		      \item $11, 15, 19$.

		            Factor each number:
		            \[
			            11 = 11,\quad 15 = 3 \cdot 5,\quad 19 = 19.
		            \]
		            The prime factors are:
		            \[
			            11: \{11\},\quad 15: \{3,5\},\quad 19: \{19\}.
		            \]
		            No prime factor is shared between any two numbers, so
		            \[
			            \gcd(11,15) = 1,\quad \gcd(11,19)=1,\quad \gcd(15,19)=1.
		            \]
		            Hence the set is pairwise relatively prime.

		            \medskip

		      \item $12, 17, 31, 37$.

		            Factor each number:
		            \[
			            12 = 2^2 \cdot 3,\quad 17,31,37 \text{ are prime.}
		            \]
		            The primes $17,31,37$ are all distinct and none of them divides $12$.
		            Therefore every pair from $\{12,17,31,37\}$ has gcd $1$, so this set is also pairwise relatively prime.

	      \end{enumerate}

	      \vspace{0.4cm}

	      %------------------------------------------------------------
  \newpage
	\item What are the greatest common divisors of each of these pairs of integers?

	      \begin{enumerate}[label=(\alph*)]

		      \item $\bigl(3^7 \cdot 5^3 \cdot 7^3\bigr)$ and $\bigl(2^{11} \cdot 3^5 \cdot 5^9 \cdot 7^3\bigr)$.
		      \item $(11 \cdot 13 \cdot 17)$ and $\bigl(2^9 \cdot 3^7 \cdot 5^5 \cdot 7^3\bigr)$.
		      \item $(41 \cdot 43 \cdot 53)$ and $(41 \cdot 43 \cdot 53)$.
		      \item $\bigl(3^{13} \cdot 5^{17}\bigr)$ and $\bigl(2^{12} \cdot 7^{21}\bigr)$.

	      \end{enumerate}

	      \textbf{Solution.}

	      Recall: if
	      \[
		      a = \prod p_i^{\alpha_i},\quad b = \prod p_i^{\beta_i}
	      \]
	      are prime factorisations over the same set of primes $p_i$, then
	      \[
		      \gcd(a,b) = \prod p_i^{\min(\alpha_i,\beta_i)}.
	      \]

	      \begin{enumerate}[label=(\alph*)]

		      \item
		            \[
			            a = 3^7 \cdot 5^3 \cdot 7^3,\quad
			            b = 2^{11} \cdot 3^5 \cdot 5^9 \cdot 7^3.
		            \]
		            Take the minimum exponents for each prime present in both:
		            \[
			            \gcd(a,b) = 3^{\min(7,5)} \cdot 5^{\min(3,9)} \cdot 7^{\min(3,3)}
			            = 3^5 \cdot 5^3 \cdot 7^3.
		            \]
		            Numerically,
		            \[
			            3^5 \cdot 5^3 \cdot 7^3 = 10{,}418{,}625.
		            \]

		            \medskip

		      \item
		            \[
			            a = 11 \cdot 13 \cdot 17,\quad
			            b = 2^9 \cdot 3^7 \cdot 5^5 \cdot 7^3.
		            \]
		            The primes in $a$ are $11,13,17$, while the primes in $b$ are $2,3,5,7$.
		            There is no common prime factor, so
		            \[
			            \gcd(a,b) = 1.
		            \]

		            \medskip

		      \item
		            \[
			            a = 41 \cdot 43 \cdot 53,\quad
			            b = 41 \cdot 43 \cdot 53.
		            \]
		            Here $a$ and $b$ are the same number, so
		            \[
			            \gcd(a,b) = 41 \cdot 43 \cdot 53.
		            \]
		            (Numerically, this is $93{,}439$.)

		            \medskip

		      \item
		            \[
			            a = 3^{13} \cdot 5^{17},\quad
			            b = 2^{12} \cdot 7^{21}.
		            \]
		            The primes in $a$ are $3$ and $5$; the primes in $b$ are $2$ and $7$.
		            There is no overlap, so
		            \[
			            \gcd(a,b) = 1.
		            \]

	      \end{enumerate}

	      \vspace{0.4cm}

	      %------------------------------------------------------------
	\item Find $\gcd(1000, 625)$ and $\operatorname{lcm}(1000, 625)$ and verify that
	      \[
		      \gcd(1000, 625) \cdot \operatorname{lcm}(1000, 625) = 1000 \cdot 625.
	      \]

	      \textbf{Solution.}

	      First find the prime factorisations:
	      \[
		      1000 = 10^3 = (2 \cdot 5)^3 = 2^3 \cdot 5^3,
		      \qquad
		      625 = 5^4.
	      \]

	      The gcd takes the minimum power of each prime:
	      \[
		      \gcd(1000, 625) = 2^{\min(3,0)} \cdot 5^{\min(3,4)} = 2^0 \cdot 5^3 = 125.
	      \]

	      The lcm takes the maximum power of each prime:
	      \[
		      \operatorname{lcm}(1000, 625) = 2^{\max(3,0)} \cdot 5^{\max(3,4)}
		      = 2^3 \cdot 5^4 = 8 \cdot 625 = 5000.
	      \]

	      Now verify the product identity:
	      \[
		      \gcd(1000, 625) \cdot \operatorname{lcm}(1000, 625)
		      = 125 \cdot 5000 = 625{,}000,
	      \]
	      and
	      \[
		      1000 \cdot 625 = 625{,}000.
	      \]
	      So the equality holds.

	      \vspace{0.4cm}

	      %------------------------------------------------------------
	\item Use the Euclidean algorithm to find:
	      \begin{enumerate}[label=(\alph*)]
		      \item $\gcd(111, 201)$;
		      \item $\gcd(1001, 1331)$;
		      \item $\gcd(1000, 5040)$.
	      \end{enumerate}

	      \textbf{Solution.}

	      \begin{enumerate}[label=(\alph*)]

		      \item $\gcd(111, 201)$.

		            Apply the Euclidean algorithm:
		            \[
			            201 = 1 \cdot 111 + 90,
		            \]
		            \[
			            111 = 1 \cdot 90 + 21,
		            \]
		            \[
			            90 = 4 \cdot 21 + 6,
		            \]
		            \[
			            21 = 3 \cdot 6 + 3,
		            \]
		            \[
			            6 = 2 \cdot 3 + 0.
		            \]
		            The last non-zero remainder is $3$, so
		            \[
			            \gcd(111, 201) = 3.
		            \]

		            \medskip

		      \item $\gcd(1001, 1331)$.

		            \[
			            1331 = 1 \cdot 1001 + 330,
		            \]
		            \[
			            1001 = 3 \cdot 330 + 11,
		            \]
		            \[
			            330 = 30 \cdot 11 + 0.
		            \]
		            The last non-zero remainder is $11$, so
		            \[
			            \gcd(1001, 1331) = 11.
		            \]

		            \medskip

		      \item $\gcd(1000, 5040)$.

		            \[
			            5040 = 5 \cdot 1000 + 40,
		            \]
		            \[
			            1000 = 25 \cdot 40 + 0.
		            \]
		            The last non-zero remainder is $40$, so
		            \[
			            \gcd(1000, 5040) = 40.
		            \]

	      \end{enumerate}

	      \vspace{0.4cm}

	      %------------------------------------------------------------
	\item Express the greatest common divisor of each of these pairs of integers $a,b$ as
	      \[
		      sa + tb
	      \]
	      for some integers $s$ and $t$.

	      \begin{enumerate}[label=(\alph*)]
		      \item $117, 213$
		      \item $3454, 4666$
	      \end{enumerate}

	      \textbf{Solution.}

	      \begin{enumerate}[label=(\alph*)]

		      \item $a = 117,\ b = 213$.

		            From the Euclidean algorithm:
		            \[
			            213 = 1 \cdot 117 + 96,
		            \]
		            \[
			            117 = 1 \cdot 96 + 21,
		            \]
		            \[
			            96 = 4 \cdot 21 + 12,
		            \]
		            \[
			            21 = 1 \cdot 12 + 9,
		            \]
		            \[
			            12 = 1 \cdot 9 + 3,
		            \]
		            \[
			            9 = 3 \cdot 3 + 0.
		            \]
		            Thus $\gcd(117,213) = 3$.

		            Now work backwards to express $3$ in terms of $117$ and $213$.

		            From $12 = 9 + 3$ we get
		            \[
			            3 = 12 - 9.
		            \]
		            From $21 = 12 + 9$ we get $9 = 21 - 12$, so
		            \[
			            3 = 12 - (21 - 12) = 2 \cdot 12 - 21.
		            \]
		            From $96 = 4 \cdot 21 + 12$ we get $12 = 96 - 4 \cdot 21$, hence
		            \[
			            3 = 2(96 - 4 \cdot 21) - 21
			            = 2 \cdot 96 - 9 \cdot 21.
		            \]
		            From $117 = 96 + 21$ we get $21 = 117 - 96$, so
		            \[
			            3 = 2 \cdot 96 - 9(117 - 96)
			            = 11 \cdot 96 - 9 \cdot 117.
		            \]
		            From $213 = 117 + 96$ we get $96 = 213 - 117$, giving
		            \[
			            3 = 11(213 - 117) - 9 \cdot 117
			            = 11 \cdot 213 - 20 \cdot 117.
		            \]

		            Thus
		            \[
			            3 = (-20) \cdot 117 + 11 \cdot 213,
		            \]
		            so we can take
		            \[
			            s = -20,\quad t = 11.
		            \]

		            \medskip

		      \item $a = 3454,\ b = 4666$.

		            First use the Euclidean algorithm to find the gcd:
		            \[
			            4666 = 1 \cdot 3454 + 1212,
		            \]
		            \[
			            3454 = 2 \cdot 1212 + 1030,
		            \]
		            \[
			            1212 = 1 \cdot 1030 + 182,
		            \]
		            \[
			            1030 = 5 \cdot 182 + 120,
		            \]
		            \[
			            182 = 1 \cdot 120 + 62,
		            \]
		            \[
			            120 = 1 \cdot 62 + 58,
		            \]
		            \[
			            62 = 1 \cdot 58 + 4,
		            \]
		            \[
			            58 = 14 \cdot 4 + 2,
		            \]
		            \[
			            4 = 2 \cdot 2 + 0.
		            \]
		            So $\gcd(3454,4666) = 2$.

		            Now work backwards to express $2$ as a linear combination of $3454$ and $4666$.

		            From $58 = 14 \cdot 4 + 2$ we have
		            \[
			            2 = 58 - 14 \cdot 4.
		            \]
		            From $62 = 58 + 4$ we have $4 = 62 - 58$, hence
		            \[
			            2 = 58 - 14(62 - 58) = 15 \cdot 58 - 14 \cdot 62.
		            \]
		            From $120 = 62 + 58$ we have $58 = 120 - 62$, so
		            \[
			            2 = 15(120 - 62) - 14 \cdot 62 = 15 \cdot 120 - 29 \cdot 62.
		            \]
		            From $182 = 120 + 62$ we have $62 = 182 - 120$, hence
		            \[
			            2 = 15 \cdot 120 - 29(182 - 120)
			            = 44 \cdot 120 - 29 \cdot 182.
		            \]
		            From $1030 = 5 \cdot 182 + 120$ we have $120 = 1030 - 5 \cdot 182$, giving
		            \[
			            2 = 44(1030 - 5 \cdot 182) - 29 \cdot 182
			            = 44 \cdot 1030 - 249 \cdot 182.
		            \]
		            From $1212 = 1030 + 182$ we have $182 = 1212 - 1030$, so
		            \[
			            2 = 44 \cdot 1030 - 249(1212 - 1030)
			            = 293 \cdot 1030 - 249 \cdot 1212.
		            \]
		            From $3454 = 2 \cdot 1212 + 1030$ we have $1030 = 3454 - 2 \cdot 1212$, hence
		            \[
			            2 = 293(3454 - 2 \cdot 1212) - 249 \cdot 1212
			            = 293 \cdot 3454 - 835 \cdot 1212.
		            \]
		            Finally, from $4666 = 3454 + 1212$ we have $1212 = 4666 - 3454$, so
		            \[
			            2 = 293 \cdot 3454 - 835(4666 - 3454)
			            = (293 + 835) \cdot 3454 - 835 \cdot 4666
			            = 1128 \cdot 3454 - 835 \cdot 4666.
		            \]

		            Thus
		            \[
			            2 = 1128 \cdot 3454 + (-835) \cdot 4666,
		            \]
		            so we can take
		            \[
			            s = 1128,\quad t = -835.
		            \]

	      \end{enumerate}

\end{enumerate}

\end{document}
