\documentclass[12pt]{scrartcl}

\usepackage{amsmath,amssymb}
\usepackage{fullpage}
\usepackage{enumitem}
\usepackage{hyperref}

\setlength{\parindent}{0pt}

\begin{document}

\begin{center}
	\hrule
	\vspace{0.4cm}
	{\textbf{\large INF1003 Tutorial 1}}\\[0.2cm]
\end{center}

\textbf{Name:} Timothy Chia\hspace{\fill} \textbf{Topic:} Sequences and Summation\\ 
\textbf{Student ID:} 2501530 \hspace{\fill} \textbf{Due Date:} 7th September 2025, 10:00 PM\\

\hrule

\begin{enumerate}[label=\textbf{\arabic*.}]

	%------------------------------------------------------------
	\item \textbf{What are the values of these sums?}
	      \begin{enumerate}[label=(\alph*)]

		      \item $\displaystyle \sum_{k=1}^{5} (k+1)$.

		            \textbf{Solution.}
		            \[
			            \sum_{k=1}^{5} (k+1)
			            = \sum_{k=1}^{5} k \;+\; \sum_{k=1}^{5} 1.
		            \]
		            First,
		            \[
			            \sum_{k=1}^{5} k = 1 + 2 + 3 + 4 + 5 = 15.
		            \]
		            Next,
		            \[
			            \sum_{k=1}^{5} 1 = 1 + 1 + 1 + 1 + 1 = 5.
		            \]
		            Therefore,
		            \[
			            \sum_{k=1}^{5} (k+1) = 15 + 5 = 20.
		            \]

		            \medskip

		      \item $\displaystyle \sum_{j=0}^{4} (-2)^j$.

		            \textbf{Solution.}
		            Write out the terms:
		            \[
			            \sum_{j=0}^{4} (-2)^j
			            = (-2)^0 + (-2)^1 + (-2)^2 + (-2)^3 + (-2)^4.
		            \]
		            Compute each power:
		            \[
			            (-2)^0 = 1,\quad
			            (-2)^1 = -2,\quad
			            (-2)^2 = 4,\quad
			            (-2)^3 = -8,\quad
			            (-2)^4 = 16.
		            \]
		            Hence
		            \[
			            \sum_{j=0}^{4} (-2)^j
			            = 1 - 2 + 4 - 8 + 16.
		            \]
		            Add step by step:
		            \[
			            1 - 2 = -1,\quad
			            -1 + 4 = 3,\quad
			            3 - 8 = -5,\quad
			            -5 + 16 = 11.
		            \]
		            So
		            \[
			            \sum_{j=0}^{4} (-2)^j = 11.
		            \]

		            \medskip

		      \item $\displaystyle \sum_{i=1}^{10} 3$.

		            \textbf{Solution.}
		            Each term in the sum is $3$, and there are $10$ terms, so
		            \[
			            \sum_{i=1}^{10} 3 = 3 + 3 + \cdots + 3 \quad (10\ \text{terms})
			            = 10 \times 3 = 30.
		            \]

		            \medskip

		      \item $\displaystyle \sum_{j=0}^{8} \bigl(2^{j+1} - 2^j\bigr)$.

		            \textbf{Solution.}
		            Factor each term:
		            \[
			            2^{j+1} - 2^j = 2^j(2 - 1) = 2^j.
		            \]
		            Therefore the sum simplifies to
		            \[
			            \sum_{j=0}^{8} \bigl(2^{j+1} - 2^j\bigr)
			            = \sum_{j=0}^{8} 2^j.
		            \]
		            This is a geometric series with first term $1$ and common ratio $2$.
		            With $9$ terms ($j=0,1,\dots,8$), the sum is
		            \[
			            \sum_{j=0}^{8} 2^j = \frac{2^{9} - 1}{2 - 1} = 2^9 - 1 = 512 - 1 = 511.
		            \]

	      \end{enumerate}

	      \vspace{0.4cm}

	      %------------------------------------------------------------
	\item \textbf{Find the next four terms of each of the following sequences after the initial conditions, defined by each of these recurrence relations.}
	      \begin{enumerate}[label=(\alph*)]

		      \item $a_n = a_{n-1}^2,\quad a_1 = 2$.

		            \textbf{Solution.}
		            We repeatedly square the previous term.

		            For $n=2$:
		            \[
			            a_2 = a_1^2 = 2^2 = 4.
		            \]
		            For $n=3$:
		            \[
			            a_3 = a_2^2 = 4^2 = 16.
		            \]
		            For $n=4$:
		            \[
			            a_4 = a_3^2 = 16^2 = 256.
		            \]
		            For $n=5$:
		            \[
			            a_5 = a_4^2 = 256^2 = 65{,}536.
		            \]
		            Thus the next four terms are
		            \[
			            a_2 = 4,\quad a_3 = 16,\quad a_4 = 256,\quad a_5 = 65{,}536.
		            \]

		            \medskip

		      \item $a_n = n a_{n-1} + n^2 a_{n-2},\quad a_1 = 1,\ a_2 = 1$.

		            \textbf{Solution.}
		            We compute successively for $n = 3,4,5,6$.

		            For $n=3$:
		            \[
			            a_3 = 3a_2 + 3^2 a_1
			            = 3(1) + 9(1)
			            = 3 + 9
			            = 12.
		            \]

		            For $n=4$:
		            \[
			            a_4 = 4a_3 + 4^2 a_2
			            = 4(12) + 16(1)
			            = 48 + 16
			            = 64.
		            \]

		            For $n=5$:
		            \[
			            a_5 = 5a_4 + 5^2 a_3
			            = 5(64) + 25(12)
			            = 320 + 300
			            = 620.
		            \]

		            For $n=6$:
		            \[
			            a_6 = 6a_5 + 6^2 a_4
			            = 6(620) + 36(64)
			            = 3720 + 2304
			            = 6024.
		            \]

		            Hence the next four terms are
		            \[
			            a_3 = 12,\quad a_4 = 64,\quad a_5 = 620,\quad a_6 = 6024.
		            \]

		            \medskip

		      \item $a_n = a_{n-1} + a_{n-3},\quad a_1 = 1,\ a_2 = 2,\ a_3 = 0$.

		            \textbf{Solution.}
		            For $n=4$:
		            \[
			            a_4 = a_3 + a_1 = 0 + 1 = 1.
		            \]
		            For $n=5$:
		            \[
			            a_5 = a_4 + a_2 = 1 + 2 = 3.
		            \]
		            For $n=6$:
		            \[
			            a_6 = a_5 + a_3 = 3 + 0 = 3.
		            \]
		            For $n=7$:
		            \[
			            a_7 = a_6 + a_4 = 3 + 1 = 4.
		            \]

		            Thus the next four terms are
		            \[
			            a_4 = 1,\quad a_5 = 3,\quad a_6 = 3,\quad a_7 = 4.
		            \]

	      \end{enumerate}

	      \vspace{0.4cm}

	      %------------------------------------------------------------
	\item Suppose that \$1,000 is invested in an account that pays compound interest at a fixed rate of $7\%$ annually. How much is there in the account after $4$ years?

	      \textbf{Solution.}
	      With annual compounding, the amount after $n$ years at interest rate $r$ is
	      \[
		      A = P(1+r)^n,
	      \]
	      where $P$ is the principal.

	      Here $P = 1000$, $r = 0.07$, $n = 4$:
	      \[
		      A = 1000(1.07)^4.
	      \]
	      Compute step by step:
	      \[
		      (1.07)^2 = 1.1449,\quad
		      (1.07)^4 = (1.07)^2 \cdot (1.07)^2 \approx 1.1449^2 \approx 1.310796\ldots
	      \]
	      So
	      \[
		      A \approx 1000 \times 1.310796 \approx 1310.80.
	      \]
	      Therefore, after $4$ years there is approximately
	      \[
		      \$1{,}310.80
	      \]
	      in the account.

	      \vspace{0.4cm}

	      %------------------------------------------------------------
	\item Find the sum of the first $n$ terms of an arithmetic series whose first term is $1$ and whose common difference is $5$.

	      \textbf{Solution.}
	      An arithmetic series with first term $a_1$ and common difference $d$ has $k$-th term
	      \[
		      a_k = a_1 + (k-1)d.
	      \]
	      Here $a_1 = 1$ and $d = 5$, so
	      \[
		      a_k = 1 + 5(k-1).
	      \]

	      The sum of the first $n$ terms is
	      \[
		      S_n = \frac{n}{2}\bigl(a_1 + a_n\bigr).
	      \]
	      We first find $a_n$:
	      \[
		      a_n = 1 + 5(n-1) = 1 + 5n - 5 = 5n - 4.
	      \]
	      Thus
	      \[
		      S_n = \frac{n}{2}\bigl(1 + (5n - 4)\bigr)
		      = \frac{n}{2}(5n - 3).
	      \]

	      So the sum of the first $n$ terms is
	      \[
		      S_n = \frac{n(5n - 3)}{2}.
	      \]

	      \vspace{0.4cm}

	      %------------------------------------------------------------
	\item An arithmetic progression has its seventh term equal to $7$ and the sum of its first $10$ terms is $60$. Find the first term and the common difference.

	      \textbf{Solution.}
	      Let the first term be $a$ and the common difference be $d$.

	      The $n$-th term is
	      \[
		      T_n = a + (n-1)d.
	      \]
	      Given that $T_7 = 7$:
	      \[
		      a + 6d = 7. \tag{1}
	      \]

	      The sum of the first $10$ terms is
	      \[
		      S_{10} = \frac{10}{2}\bigl(2a + 9d\bigr) = 5(2a + 9d).
	      \]
	      We are told $S_{10} = 60$, so
	      \[
		      5(2a + 9d) = 60
		      \quad\Rightarrow\quad
		      2a + 9d = 12. \tag{2}
	      \]

	      Now solve the simultaneous equations (1) and (2).

	      From (1):
	      \[
		      a = 7 - 6d.
	      \]
	      Substitute into (2):
	      \[
		      2(7 - 6d) + 9d = 12
		      \quad\Rightarrow\quad
		      14 - 12d + 9d = 12
		      \quad\Rightarrow\quad
		      14 - 3d = 12.
	      \]
	      Hence
	      \[
		      -3d = 12 - 14 = -2
		      \quad\Rightarrow\quad
		      d = \frac{2}{3}.
	      \]

	      Then
	      \[
		      a = 7 - 6d = 7 - 6\cdot\frac{2}{3} = 7 - 4 = 3.
	      \]

	      So the first term and common difference are
	      \[
		      a = 3,\qquad d = \frac{2}{3}.
	      \]

	      \vspace{0.4cm}

	      %------------------------------------------------------------
	\item Find an expression for
	      \[
		      2 + 2(3) + 2(3^2) + \cdots + 2(3^n).
	      \]

	      \textbf{Solution.}
	      Factor out $2$:
	      \[
		      2 + 2(3) + 2(3^2) + \cdots + 2(3^n)
		      = 2\bigl(1 + 3 + 3^2 + \cdots + 3^n\bigr).
	      \]
	      The sum inside the brackets is a geometric series with first term $1$, common ratio $3$, and $(n+1)$ terms.

	      The sum of the first $(n+1)$ terms of a geometric series is
	      \[
		      1 + 3 + 3^2 + \cdots + 3^n
		      = \frac{3^{n+1} - 1}{3 - 1}
		      = \frac{3^{n+1} - 1}{2}.
	      \]
	      Therefore
	      \[
		      2\bigl(1 + 3 + 3^2 + \cdots + 3^n\bigr)
		      = 2 \cdot \frac{3^{n+1} - 1}{2}
			      = 3^{n+1} - 1.
	      \]

	      So the required expression is
	      \[
		      3^{n+1} - 1.
	      \]

	      \vspace{0.4cm}

	      %------------------------------------------------------------
	\item \textbf{What is the value of each of the following sums?}
	      \begin{enumerate}[label=(\alph*)]

		      \item $\displaystyle \sum_{j=0}^{8} 3 \cdot 2^j$.

		            \textbf{Solution.}
		            Factor out the constant $3$:
		            \[
			            \sum_{j=0}^{8} 3 \cdot 2^j
			            = 3 \sum_{j=0}^{8} 2^j.
		            \]
		            The inner sum is geometric with first term $1$, common ratio $2$, and $9$ terms:
		            \[
			            \sum_{j=0}^{8} 2^j
			            = \frac{2^{9} - 1}{2 - 1}
			            = 512 - 1
			            = 511.
		            \]
		            Hence
		            \[
			            \sum_{j=0}^{8} 3 \cdot 2^j
			            = 3 \cdot 511
			            = 1533.
		            \]

		            \medskip

		      \item $\displaystyle \sum_{j=2}^{8} (-3)^j$.

		            \textbf{Solution.}
		            This is a geometric series with first term
		            \[
			            a_1 = (-3)^2 = 9,
		            \]
		            common ratio
		            \[
			            r = -3,
		            \]
		            and the indices run from $j=2$ to $j=8$, so there are $8 - 2 + 1 = 7$ terms.

		            The sum of $m$ terms of a geometric series is
		            \[
			            S_m = a_1\,\frac{1 - r^m}{1 - r}.
		            \]
		            Here $m=7$, so
		            \[
			            \sum_{j=2}^{8} (-3)^j
			            = 9 \cdot \frac{1 - (-3)^7}{1 - (-3)}.
		            \]
		            Compute:
		            \[
			            (-3)^7 = -2187,
			            \quad 1 - (-3)^7 = 1 - (-2187) = 1 + 2187 = 2188,
		            \]
		            and
		            \[
			            1 - (-3) = 1 + 3 = 4.
		            \]
		            Thus
		            \[
			            \sum_{j=2}^{8} (-3)^j
			            = 9 \cdot \frac{2188}{4}
			            = \frac{9 \cdot 2188}{4}
			            = \frac{19692}{4}
			            = 4923.
		            \]

		            \medskip

		      \item $\displaystyle \sum_{j=4}^{10} (2 + 3j)$.

		            \textbf{Solution.}
		            Split the sum:
		            \[
			            \sum_{j=4}^{10} (2 + 3j)
			            = \sum_{j=4}^{10} 2 + \sum_{j=4}^{10} 3j.
		            \]

		            There are $10 - 4 + 1 = 7$ terms, so
		            \[
			            \sum_{j=4}^{10} 2 = 2 \times 7 = 14.
		            \]

		            For the second part,
		            \[
			            \sum_{j=4}^{10} 3j = 3 \sum_{j=4}^{10} j.
		            \]
		            Now
		            \[
			            \sum_{j=1}^{10} j = \frac{10\cdot11}{2} = 55,\quad
			            \sum_{j=1}^{3} j = \frac{3\cdot4}{2} = 6,
		            \]
		            so
		            \[
			            \sum_{j=4}^{10} j = \sum_{j=1}^{10} j - \sum_{j=1}^{3} j = 55 - 6 = 49.
		            \]
		            Hence
		            \[
			            \sum_{j=4}^{10} 3j = 3 \cdot 49 = 147.
		            \]

		            Putting the parts together:
		            \[
			            \sum_{j=4}^{10} (2 + 3j)
			            = 14 + 147 = 161.
		            \]

	      \end{enumerate}

	      \vspace{0.4cm}

	      %------------------------------------------------------------
	\item Suppose $a_n = \dfrac{1}{2^{2n}}$. Find the limit, as $n \to \infty$, of
	      \[
		      S_n = a_0 + a_1 + \cdots + a_{n-1}.
	      \]

	      \textbf{Solution.}
	      First write down the first few terms:
	      \[
		      a_0 = \frac{1}{2^{0}} = 1,\quad
		      a_1 = \frac{1}{2^{2}} = \frac{1}{4},\quad
		      a_2 = \frac{1}{2^{4}} = \frac{1}{16},\ \dots
	      \]
	      So
	      \[
		      S_n = 1 + \frac{1}{4} + \frac{1}{16} + \cdots + \frac{1}{2^{2(n-1)}}.
	      \]

	      This is a geometric series with
	      \[
		      \text{first term } a = 1,\qquad
		      \text{common ratio } r = \frac{1}{4}.
	      \]
	      There are $n$ terms in $S_n$ (from $a_0$ to $a_{n-1}$). The sum of the first $n$ terms is
	      \[
		      S_n = a\,\frac{1 - r^n}{1 - r}
		      = \frac{1 - \left(\frac{1}{4}\right)^n}{1 - \frac{1}{4}}
		      = \frac{1 - \left(\frac{1}{4}\right)^n}{\frac{3}{4}}
		      = \frac{4}{3}\left(1 - \left(\frac{1}{4}\right)^n\right).
	      \]
	      As $n \to \infty$, $\left(\frac{1}{4}\right)^n \to 0$, so
	      \[
		      \lim_{n \to \infty} S_n = \frac{4}{3} (1 - 0) = \frac{4}{3}.
	      \]

	      Therefore
	      \[
		      \boxed{\displaystyle \lim_{n \to \infty} S_n = \frac{4}{3}.}
	      \]

\end{enumerate}

\end{document}
