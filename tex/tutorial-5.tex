\documentclass[12pt]{scrartcl}

\usepackage{amsmath,amssymb}
\usepackage{fullpage}
\usepackage{enumitem}
\usepackage{hyperref}

\setlength{\parindent}{0pt}

\begin{document}

\begin{center}
    \hrule
    \vspace{0.4cm}
    {\textbf{\large INF1003 Tutorial 5}}\\[0.2cm]
\end{center}

\textbf{Name:} Timothy Chia\hspace{\fill} \textbf{Topic:} Predicate Logic\\
\textbf{Student ID:} 2501530 \hspace{\fill} \textbf{Due Date:} 5/10/2025 10:00PM\\

\hrule

\begin{enumerate}[label=\textbf{\arabic*.}]

%------------------------------------------------------------
\item 
Let $\mathrm{Prime}(n)$ be the predicate ``$n$ is a prime number'' and $\mathrm{In}(n,a,b)$ be the
predicate $a \leq n \leq b$, where $n,a,b$ range over all integers.
For each statement below, determine whether it is \emph{True}, \emph{False}, or \emph{Neither},
and give a brief explanation.

\begin{enumerate}[label=(\alph*)]

    \item $\neg \mathrm{Prime}(10) \lor \mathrm{In}(10,5,20)$.

    \textbf{Solution.}
    \begin{itemize}
        \item $10$ is not prime, so $\mathrm{Prime}(10)$ is false and $\neg \mathrm{Prime}(10)$ is true.
        \item $10$ lies between $5$ and $20$, so $\mathrm{In}(10,5,20)$ is also true.
        \item The whole statement is $T \lor T$, which is true.
    \end{itemize}
    Hence the statement is \textbf{True}.

    \item $\exists n\, \mathrm{Prime}(n)$.

    \textbf{Solution.}
    There are many prime integers (for example, $2$, $3$, $5$, \dots).  
    Choosing $n=2$ gives $\mathrm{Prime}(2)$ true, so the existential statement is satisfied.  
    Therefore this statement is \textbf{True}.

    \item $\exists n\, \neg \mathrm{Prime}(n)$.

    \textbf{Solution.}
    There are integers that are not prime, for example $1$, $4$, $6$, \dots.  
    Taking $n=4$ gives $\neg \mathrm{Prime}(4)$ true.  
    So the existential statement holds and is \textbf{True}.

    \item $\forall n\, \mathrm{Prime}(n)$.

    \textbf{Solution.}
    This says every integer is prime.  
    But $4$ (or $1$, or $10$, etc.) is not prime, so there is a counterexample.
    Because at least one $n$ makes $\mathrm{Prime}(n)$ false, the universal statement is false.  
    Hence it is \textbf{False}.

    \item $\forall n\, \neg \mathrm{Prime}(n)$.

    \textbf{Solution.}
    This says no integer is prime.  
    However $2$ is prime, so for $n=2$ we have $\neg \mathrm{Prime}(2)$ false.
    Thus the universal statement fails and is \textbf{False}.

    \item $\neg \forall n\, \mathrm{Prime}(n)$.

    \textbf{Solution.}
    Using a standard equivalence:
    \[
        \neg \forall n\, \mathrm{Prime}(n) \equiv \exists n\, \neg \mathrm{Prime}(n).
    \]
    In part (c) we saw that $\exists n\, \neg \mathrm{Prime}(n)$ is true (e.g.\ $n=4$).
    Therefore this statement is \textbf{True}.

    \item $\forall n\, (\mathrm{In}(n,1,3) \rightarrow \mathrm{Prime}(n))$.

    \textbf{Solution.}
    The antecedent $\mathrm{In}(n,1,3)$ is true exactly when $n=1,2,$ or $3$.  
    We check these cases:
    \begin{itemize}
        \item $n=1$: $\mathrm{In}(1,1,3)$ is true, but $1$ is not prime, so $\mathrm{Prime}(1)$ is false.
              Hence the implication is $T \rightarrow F$, which is false.
    \end{itemize}
    Because the universal quantifier requires the implication to hold for \emph{every} $n$,
    the single counterexample $n=1$ makes the whole statement \textbf{False}.

    \item $\forall n\, (\mathrm{In}(n,8,10) \rightarrow \mathrm{Prime}(n))$.

    \textbf{Solution.}
    Here $\mathrm{In}(n,8,10)$ is true for $n=8,9,10$.
    \begin{itemize}
        \item $n=9$: $\mathrm{In}(9,8,10)$ is true, but $9$ is not prime ($9=3\cdot 3$),
              so $\mathrm{Prime}(9)$ is false.
              The implication is therefore $T \rightarrow F$, which is false.
    \end{itemize}
    Hence the universal statement is \textbf{False}.

    \item $\forall n\,(\mathrm{In}(n,a,b) \rightarrow \neg \mathrm{Prime}(n))$, where $a$ and $b$
          are integers smaller than $10$.

    \textbf{Solution.}
    Here $a$ and $b$ are fixed integers less than $10$, but we are \emph{not} told which ones.
    \begin{itemize}
        \item If the interval $[a,b]$ happens to contain a prime (for example $[2,4]$),
              then the statement is false (because some $n$ in $[a,b]$ will have $\mathrm{Prime}(n)$ true).
        \item If $[a,b]$ is chosen to contain no primes (for example $[1,1]$ or $[8,9]$),
              then the statement is true.
    \end{itemize}
    Since the truth value depends on the particular (unspecified) values of $a$ and $b$,
    we cannot classify it as always true or always false.  
    Therefore the answer is \textbf{Neither}.

    \item $\exists n\, \bigl(\mathrm{Prime}(n) \rightarrow \mathrm{In}(n,30,40)\bigr)$.

    \textbf{Solution.}
    First interpret the formula carefully:
    \[
        \exists n\, (\mathrm{Prime}(n) \rightarrow \mathrm{In}(n,30,40)).
    \]
    For a \emph{fixed} integer $n$, the implication $\mathrm{Prime}(n) \rightarrow \mathrm{In}(n,30,40)$
    is false only when $\mathrm{Prime}(n)$ is true and $\mathrm{In}(n,30,40)$ is false;
    in all other cases it is true.
    \begin{itemize}
        \item Take $n=1$. Then $\mathrm{Prime}(1)$ is false and so the implication is
              $F \rightarrow \mathrm{In}(1,30,40)$, which is true regardless of the consequent.
    \end{itemize}
    Thus there exists at least one integer $n$ (for example $n=1$) making the implication true,
    so the existential statement holds.  
    Therefore it is \textbf{True}.
\end{enumerate}

%------------------------------------------------------------
\item 
Let $\mathrm{BB}(x)$ be the statement ``$x$ plays basketball every week'', where the domain of $x$
is all ICT students. Express each logical formula in English.

\begin{enumerate}[label=(\alph*)]

    \item $\exists x\, \mathrm{BB}(x)$.

    \textbf{Solution.}
    There is at least one ICT student who plays basketball every week.

    \item $\forall x\, \mathrm{BB}(x)$.

    \textbf{Solution.}
    Every ICT student plays basketball every week.

    \item $\neg \forall x\, \mathrm{BB}(x)$.

    \textbf{Solution.}
    First note that
    \[
        \neg \forall x\, \mathrm{BB}(x) \equiv \exists x\, \neg \mathrm{BB}(x).
    \]
    So in English: Not every ICT student plays basketball every week.

    \item $\exists x\, \neg \mathrm{BB}(x)$.

    \textbf{Solution.}
    There is an ICT student who does not play basketball every week.

\end{enumerate}

%------------------------------------------------------------
\item 
For each statement, do the following:
\begin{itemize}
    \item[(I)] Express it using quantified logical expressions.
    \item[(II)] Form the negation, pushing all negations directly onto predicates (not onto quantifiers).
    \item[(III)] Express the negated statement in English.
\end{itemize}

\begin{enumerate}[label=(\alph*)]

    \item No rabbit knows calculus.

    \textbf{Solution.}
    Let the domain be all rabbits, and let $K(x)$ mean ``$x$ knows calculus''.

    \textbf{(I)} Original statement:
    \[
        \forall x\, \neg K(x)
    \]
    (equivalently $\neg \exists x\, K(x)$).

    \textbf{(II)} Negation:
    \[
        \neg \forall x\, \neg K(x) \equiv \exists x\, \neg\neg K(x) \equiv \exists x\, K(x).
    \]

    \textbf{(III)} English: There is a rabbit that knows calculus.

    \item There is a bird that can talk.

    \textbf{Solution.}
    Let the domain be all birds and $T(x)$ mean ``$x$ can talk''.

    \textbf{(I)} Original statement:
    \[
        \exists x\, T(x).
    \]

    \textbf{(II)} Negation:
    \[
        \neg \exists x\, T(x) \equiv \forall x\, \neg T(x).
    \]

    \textbf{(III)} English: No bird can talk (equivalently: all birds cannot talk).

    \item There is no one in this class who knows French and Russian.

    \textbf{Solution.}
    Let the domain be all people in this class.  
    Let $F(x)$ mean ``$x$ knows French'' and $R(x)$ mean ``$x$ knows Russian''.

    \textbf{(I)} Original statement:
    \[
        \neg \exists x\, (F(x) \land R(x))
        \quad\text{or equivalently}\quad
        \forall x\, \neg(F(x) \land R(x)).
    \]

    \textbf{(II)} Negation:
    \[
        \neg \neg \exists x\, (F(x) \land R(x))
        \equiv
        \exists x\, (F(x) \land R(x)).
    \]

    \textbf{(III)} English: There is someone in this class who knows both French and Russian.

    \item Everyone in this class is a Marvel fan.

    \textbf{Solution.}
    Let the domain be all people in this class, and let $M(x)$ mean ``$x$ is a Marvel fan''.

    \textbf{(I)} Original statement:
    \[
        \forall x\, M(x).
    \]

    \textbf{(II)} Negation:
    \[
        \neg \forall x\, M(x) \equiv \exists x\, \neg M(x).
    \]

    \textbf{(III)} English: There is someone in this class who is not a Marvel fan.

\end{enumerate}

%------------------------------------------------------------
\item 
Express each of the following statements using predicates and quantifiers.
Clearly define all predicates and variables.

\begin{enumerate}[label=(\alph*)]

    \item There is a student who has taken more than $21$ credit hours in a semester and
          received all A's.

    \textbf{Solution.}
    Let the domain of $x$ be all students, and the domain of $y$ be all modules.
    \begin{itemize}
        \item $\mathrm{Credits}(x,y)$: student $x$ has taken more than $y$ credit hours
              in a semester.
        \item $A(x,y)$: student $x$ received grade A in module $y$.
    \end{itemize}
    The statement ``received all A's'' can be modelled as ``for every module $y$,
    student $x$ received A in $y$''.
    \[
        \exists x\,\bigl(\mathrm{Credits}(x,21) \land \forall y\, A(x,y)\bigr).
    \]

    \item A passenger on an airline qualifies as an ``Elite Flyer'' if the passenger flies
          more than $25000$ miles in a year or takes more than $25$ flights during that year.

    \textbf{Solution.}
    Let the domain of $x$ be all airline passengers.
    \begin{itemize}
        \item $\mathrm{Miles}(x,y)$: passenger $x$ flies more than $y$ miles in a year.
        \item $\mathrm{Flights}(x,y)$: passenger $x$ takes more than $y$ flights in a year.
        \item $\mathrm{Elite}(x)$: passenger $x$ qualifies as an Elite Flyer.
    \end{itemize}
    The rule ``qualifies as an Elite Flyer if \dots'' becomes an implication for all passengers:
    \[
        \forall x\,\bigl((\mathrm{Miles}(x,25000) \lor \mathrm{Flights}(x,25))
        \rightarrow \mathrm{Elite}(x)\bigr).
    \]

    \item A man qualifies for the marathon if his best previous time is less than $3$ hours
          and a woman qualifies for the marathon if her best previous time is less than
          $3.5$ hours.

    \textbf{Solution.}
    Let the domain of $x$ be all people.
    \begin{itemize}
        \item $M(x)$: $x$ is a man.
        \item $W(x)$: $x$ is a woman.
        \item $Q(x)$: $x$ qualifies for the marathon.
        \item $\mathrm{Best}(x,t)$: the best previous time of $x$ is less than $t$ hours.
    \end{itemize}
    We can encode both conditions in a single formula:
    \[
        \forall x\,\bigl((M(x) \land \mathrm{Best}(x,3)) 
                \lor (W(x) \land \mathrm{Best}(x,3.5))
                \rightarrow Q(x)\bigr).
    \]
    (Equivalently, we could have written two separate implications, one for men and one for women.)

\end{enumerate}

%------------------------------------------------------------
\item 
Use the following predicates:
\[
    L(x): \text{$x$ has a laptop},\quad
    D(x): \text{$x$ has a desktop computer},\quad
    M(x): \text{$x$ uses macOS},\quad
    W(x): \text{$x$ uses Windows}.
\]
Domain~1 is ``students in this class''.  
Domain~2 is ``all people in the world''.  
You may also use an additional predicate
\[
    S(x): \text{$x$ is a student in this class}
\]
when working in Domain~2.

For each statement, give an expression in both domains.

\begin{enumerate}[label=(\alph*)]

    \item Some students have both a laptop and a desktop computer.

    \textbf{Solution.}
    \begin{itemize}
        \item \textbf{Domain 1 (students in this class):}
              \[
                \exists x\, (L(x) \land D(x)).
              \]
              Here the domain already consists of students, so we do not need $S(x)$.

        \item \textbf{Domain 2 (all people in the world):}
              We must restrict to those who are students in this class:
              \[
                \exists x\, (S(x) \land L(x) \land D(x)).
              \]
    \end{itemize}

    \item All students who use macOS have a laptop.

    \textbf{Solution.}
    \begin{itemize}
        \item \textbf{Domain 1:}
              \[
                \forall x\, (M(x) \rightarrow L(x)).
              \]
              Every (class) student using macOS has a laptop.

        \item \textbf{Domain 2:}
              We again restrict to students in this class:
              \[
                \forall x\, ((S(x) \land M(x)) \rightarrow L(x)).
              \]
    \end{itemize}

    \item Every student uses either macOS, Windows, or both.

    \textbf{Solution.}
    \begin{itemize}
        \item \textbf{Domain 1:}
              \[
                \forall x\, (M(x) \lor W(x)).
              \]
              (All students in this class use macOS, Windows, or both.)

        \item \textbf{Domain 2:}
              We restrict the universal quantifier to students in this class:
              \[
                \forall x\, (S(x) \rightarrow (M(x) \lor W(x))).
              \]
    \end{itemize}

\end{enumerate}

%------------------------------------------------------------
\item[\textbf{6.}] \textbf{[OPTIONAL]}  
Translate each logical statement into a clear English sentence.
The domain of every variable is the set of all real numbers.

\begin{enumerate}[label=(\alph*)]

    \item $\forall x\, \exists y\, (x < y)$.

    \textbf{Solution.}
    For every real number $x$, there is a real number $y$ that is larger than $x$.  
    (Every real number has a larger real number.)

    \item $\forall x\, \forall y\, (((x \ge 0) \land (y \ge 0)) \rightarrow (xy \ge 0))$.

    \textbf{Solution.}
    For all real numbers $x$ and $y$, if both $x$ and $y$ are non-negative,
    then their product $xy$ is also non-negative.

    \item $\forall x\, \forall y\, \exists z\, (x + y = z)$.

    \textbf{Solution.}
    For any two real numbers $x$ and $y$, there exists a real number $z$ equal to their sum.  
    (The sum of any two real numbers is a real number.)

\end{enumerate}

%------------------------------------------------------------
\item[\textbf{7.}] \textbf{[OPTIONAL]}  
Let $F(x,y)$ be the predicate ``$x$ and $y$ are friends'', where $x$ and $y$ range over all
students in SIT. Translate each statement into clear English (avoid using the symbols $x$ and $y$
in your English sentences).

\begin{enumerate}[label=(\alph*)]

    \item $\forall x\, \exists y\, \bigl( F(x,y) \land \forall z\, ((y \ne z) \rightarrow \neg F(x,z)) \bigr)$.

    \textbf{Solution.}
    For every student, there is some student who is their friend, and they have no other friends.
    In words: \emph{Every student in SIT has exactly one friend.}

    \item $\exists x\, \forall y\, \forall z\,
           \bigl((F(x,y) \land F(x,z) \land (y \ne z)) \rightarrow \neg F(y,z)\bigr)$.

    \textbf{Solution.}
    There is at least one student such that whenever two different students are both friends
    with that person, those two students are not friends with each other.
    In words: \emph{There is a student whose friends are not friends with one another.}

\end{enumerate}

\end{enumerate}

\end{document}
