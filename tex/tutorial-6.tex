\documentclass[12pt]{scrartcl}

\usepackage{amsmath,amssymb}
\usepackage{fullpage}
\usepackage{enumitem}
\usepackage{hyperref}

\setlength{\parindent}{0pt}

\begin{document}

\begin{center}
	\hrule
	\vspace{0.4cm}
	{\textbf{\large INF1003 Tutorial 6}}\\[0.2cm]
\end{center}

\textbf{Name:} Timothy Chia\hspace{\fill} \textbf{Topic:} Rules of Inference\\
\textbf{Student ID:} 2501530 \hspace{\fill} \textbf{Due Date:} 12/10/2025, 10:00 PM\\

\hrule

\begin{enumerate}[label=\textbf{\arabic*.}]

	%------------------------------------------------------------
	\item For each of the arguments below, define the propositions and state which rule of
	      inference is used to conclude when the premises are true. If no conclusion can be
	      drawn, explain why.

	      \begin{enumerate}[label=(\alph*)]

		      \item \emph{If I work all night on this homework, then I can answer all the exercises.
			            If I answer all the exercises, I will understand the material. Therefore, if I work all
			            night on this homework, then I will understand the material.}

		            \textbf{Propositions.}
		            \[
			            \begin{aligned}
				            p & : \text{I work all night on this homework}, \\
				            q & : \text{I can answer all the exercises},    \\
				            r & : \text{I will understand the material}.
			            \end{aligned}
		            \]

		            \textbf{Formal argument.}
		            \begin{align*}
			            1.\; & p \rightarrow q &  & \text{Premise}                                        \\
			            2.\; & q \rightarrow r &  & \text{Premise}                                        \\
			            3.\; & p \rightarrow r &  & \text{From 1 and 2 by \emph{hypothetical syllogism}.}
		            \end{align*}

		            So the rule of inference used is \textbf{hypothetical syllogism}.

		      \item \emph{If $n$ is a real number such that $n>3$, then $2n>6$.
			            Suppose that $n \ge 2$. Can you conclude $2n>6$?}

		            \textbf{Propositions.}
		            \[
			            p: n>3,\qquad
			            q: 2n>6,\qquad
			            r: n\ge 2.
			            \begin{aligned}
			            \end{aligned}
		            \]

		            \textbf{Given.}
		            \[
			            1.\; p \rightarrow q,\qquad
			            2.\; r.
		            \]

		            To use modus ponens we would need $p$ as a premise (``$n>3$''),
		            but we only know $r$ (``$n\ge 2$'').
		            From $n\ge 2$ we \emph{cannot} deduce that $n>3$.

		            For example, if $n=2.5$ then $n\ge 2$ is true, but $n>3$ is false
		            and $2n = 5$ is not greater than $6$.

		            Hence there is no valid rule of inference that lets us conclude $q$ from $1$ and $2$,
		            and the answer is: \textbf{cannot conclude}; no valid inference rule applies.

		      \item \emph{If it snows today, the university will close. The university is not
			            closed today. Therefore, it did not snow today.}

		            \textbf{Propositions.}
		            \[
			            p: \text{It snows today},\qquad
			            q: \text{The university will close today}.
		            \]

		            \textbf{Formal argument.}
		            \begin{align*}
			            1.\; & p \rightarrow q &  & \text{Premise}                               \\
			            2.\; & \neg q          &  & \text{Premise}                               \\
			            3.\; & \neg p          &  & \text{From 1 and 2 by \emph{modus tollens}.}
		            \end{align*}

		            The rule used is \textbf{modus tollens}.

		      \item \emph{I will go swimming, or I will stay in the sun too long.
			            I did not stay in the sun too long. Therefore, I went swimming.}

		            \textbf{Propositions.}
		            \[
			            p: \text{I will go swimming},\qquad
			            q: \text{I will stay in the sun too long}.
		            \]

		            \textbf{Formal argument.}
		            \begin{align*}
			            1.\; & p \lor q &  & \text{Premise}                                       \\
			            2.\; & \neg q   &  & \text{Premise}                                       \\
			            3.\; & p        &  & \text{From 1 and 2 by \emph{disjunctive syllogism}.}
		            \end{align*}

		            The rule used is \textbf{disjunctive syllogism}.

		      \item \emph{If $n$ is a real number such that $n>2$, then $n^{2}>4$.
			            Suppose that $n^{2}\le 4$, then $n\le 2$.}

		            \textbf{Propositions.}
		            \[
			            p: n>2,\qquad
			            q: n^{2}>4.
		            \]

		            \textbf{Formal argument.}
		            \begin{align*}
			            1.\; & p \rightarrow q &  & \text{Premise}                                                      \\
			            2.\; & \neg q          &  & \text{Premise (since }n^{2}\le 4\text{ means }\neg(n^{2}>4)\text{)} \\
			            3.\; & \neg p          &  & \text{From 1 and 2 by \emph{modus tollens}.}
		            \end{align*}
		            The conclusion $\neg p$ is $n\le 2$.
		            The rule used is \textbf{modus tollens}.

	      \end{enumerate}

	      %------------------------------------------------------------
	      \newpage
	\item Use rules of inference to show that if the premises (a–c) are true, then they imply the
	      conclusion \emph{``It rained''}. Define propositions clearly.

	      The premises are:
	      \begin{itemize}
		      \item[(a)] If it does not rain or if it is foggy, then the sailing race will be held and the
		            lifesaving demonstration will go on.
		      \item[(b)] If the sailing race is held, then the trophy will be awarded.
		      \item[(c)] The trophy was not awarded.
	      \end{itemize}

	      \textbf{Propositions.}
	      \[
		      \begin{aligned}
			      % p: \text{It rained} \\
			      % q: \text{It is foggy} \\
			      % r: \text{The sailing race will be held} \\
			      % s: \text{The lifesaving demonstration will go on} \\
			      % t: \text{The trophy will be awarded}.
			      p & : \text{It rained}                               \\
			      q & : \text{It is foggy}                             \\
			      r & : \text{The sailing race will be held}           \\
			      s & : \text{The lifesaving demonstration will go on} \\
			      t & : \text{The trophy will be awarded}.
		      \end{aligned}
	      \]

	      \textbf{Formal proof.}
	      \begin{align*}
		      1.\;  & (\neg p \lor q) \rightarrow (r \land s)              &                            & \text{Premise (a)}                    \\
		      2.\;  & r \rightarrow t                                      &                            & \text{Premise (b)}                    \\
		      3.\;  & \neg t                                               &                            & \text{Premise (c)}                    \\[4pt]
		      4.\;  & \neg r                                               &                            & \text{From 2 and 3 by modus tollens}  \\[4pt]
		      5.\;  & (r \land s) \rightarrow r                            &                            & \text{Tautology (simplification)}     \\
		      6.\;  & \neg(r \land s)                                      &                            & \text{From 4 and 5 by modus tollens}  \\[4pt]
		      7.\;  & \neg(r \land s) \rightarrow \neg(\neg p \lor q)
		            &                                                      & \text{Contrapositive of 1}                                         \\
		      8.\;  & \neg(\neg p \lor q)                                  &                            & \text{From 6 and 7 by modus ponens}   \\[4pt]
		      9.\;  & \neg(\neg p \lor q) \equiv (\neg\neg p \land \neg q)
		            &                                                      & \text{De Morgan’s law}                                             \\
		      10.\; & p \land \neg q                                       &                            & \text{From 8 and 9 (double negation)} \\
		      11.\; & p                                                    &                            & \text{From 10 by simplification}
	      \end{align*}

	      Thus, from the premises we can conclude that \textbf{it rained}.

	      %------------------------------------------------------------
	      \newpage
	\item Use the rule of resolution to show that premises (a–c) below imply that
	      \emph{``Alice does not get wet''}. Define propositions clearly.

	      \begin{itemize}
		      \item[(a)] It is not raining or Alice has her umbrella.
		      \item[(b)] If Alice has her umbrella, then she does not get wet.
		      \item[(c)] It is raining or Alice does not get wet.
	      \end{itemize}

	      \textbf{Propositions.}
	      \[
		      r: \text{It is raining},\quad
		      u: \text{Alice has her umbrella},\quad
		      w: \text{Alice gets wet}.
	      \]

	      \textbf{Rewrite premises in disjunctive form.}
	      \begin{align*}
		       & \text{(a)}\quad \neg r \lor u.                                  \\
		       & \text{(b)}\quad u \rightarrow \neg w \equiv \neg u \lor \neg w. \\
		       & \text{(c)}\quad r \lor \neg w.
	      \end{align*}

	      \textbf{Resolution steps.}
	      \begin{align*}
		      1.\; & \neg r \lor u      &                               & \text{From (a)} \\
		      2.\; & \neg u \lor \neg w &                               & \text{From (b)} \\
		      3.\; & r \lor \neg w      &                               & \text{From (c)} \\[4pt]
		      4.\; & \neg r \lor \neg w
		           &                    & \text{Resolve 1 and 2 on $u$}                   \\
		      5.\; & \neg w \lor \neg w
		           &                    & \text{Resolve 3 and 4 on $r$}                   \\
		      6.\; & \neg w
		           &                    & \text{Idempotent law on 5}
	      \end{align*}

	      Therefore we conclude that \textbf{Alice does not get wet}.

	      %------------------------------------------------------------
	      \newpage
	\item Consider the premises:
	      \begin{align*}
		      \text{(a)}\; & (p \lor q) \land (\neg p \rightarrow \neg q), \\
		      \text{(b)}\; & p \rightarrow r.
	      \end{align*}
	      Use known rules of inference to show that they imply the conclusion $r \lor t$.
	      Use logical equivalences where necessary.

	      \textbf{Proof.}
	      \begin{align*}
		      1.\;  & (p \lor q) \land (\neg p \rightarrow \neg q)     &                                             & \text{Premise}                               \\
		      2.\;  & p \lor q                                         &                                             & \text{From 1 by simplification}              \\
		      3.\;  & \neg p \rightarrow \neg q                        &                                             & \text{From 1 by simplification}              \\[4pt]
		      4.\;  & \neg p \rightarrow \neg q \equiv q \rightarrow p
		            &                                                  & \text{Contrapositive / logical equivalence}                                                \\
		      5.\;  & q \rightarrow p \equiv \neg q \lor p
		            &                                                  & \text{Implication equivalence}                                                             \\[4pt]
		      6.\;  & p \lor q                                         &                                             & \text{From 2}                                \\
		      7.\;  & \neg q \lor p                                    &                                             & \text{From 5}                                \\
		      8.\;  & p \lor p                                         &                                             & \text{Resolution on 6 and 7 (eliminate $q$)} \\
		      9.\;  & p                                                &                                             & \text{Idempotent law on 8}                   \\[4pt]
		      10.\; & p \rightarrow r                                  &                                             & \text{Premise (b)}                           \\
		      11.\; & r                                                &                                             & \text{From 9 and 10 by modus ponens}         \\
		      12.\; & r \lor t                                         &                                             & \text{From 11 by addition}
	      \end{align*}

	      Hence the premises imply the conclusion $r \lor t$.

	      %------------------------------------------------------------
	\item Consider the premises:
	      \begin{align*}
		      \text{(a)}\; & (p \rightarrow q) \land (r \rightarrow s), \\
		      \text{(b)}\; & p,                                         \\
		      \text{(c)}\; & \neg s.
	      \end{align*}
	      Use rules of inference and logical equivalences to show that they imply $q \land \neg r$.

	      \textbf{Proof.}
	      \begin{align*}
		      1.\; & (p \rightarrow q) \land (r \rightarrow s) &  & \text{Premise}                       \\
		      2.\; & p \rightarrow q                           &  & \text{From 1 by simplification}      \\
		      3.\; & r \rightarrow s                           &  & \text{From 1 by simplification}      \\
		      4.\; & p                                         &  & \text{Premise (b)}                   \\
		      5.\; & \neg s                                    &  & \text{Premise (c)}                   \\[4pt]
		      6.\; & q                                         &  & \text{From 2 and 4 by modus ponens}  \\
		      7.\; & \neg r                                    &  & \text{From 3 and 5 by modus tollens} \\
		      8.\; & q \land \neg r                            &  & \text{From 6 and 7 by conjunction}
	      \end{align*}

	      Thus the premises imply $q \land \neg r$.

	      %------------------------------------------------------------
	\item For each of the arguments (a–d) below, determine whether the premises imply the
	      conclusion. For valid arguments, justify using predicates and rules of inference.
	      For invalid arguments, explain why (for example, by giving a counterexample).

	      \begin{enumerate}[label=(\alph*)]

		      \item \emph{``Every student in this class passed the first class test'' and
			            ``Alice is a student in this class'' imply the conclusion
			            ``Alice passed the first class test''.}

		            \textbf{Predicates.}
		            Domain: all people.
		            \[
			            S(x): x \text{ is a student in this class},\qquad
			            P(x): x \text{ passed the first class test}.
		            \]

		            \textbf{Formal argument.}
		            \begin{align*}
			            1.\; & \forall x\,(S(x) \rightarrow P(x))          &                                          & \text{Premise} \\
			            2.\; & S(\text{Alice})                             &                                          & \text{Premise} \\
			            3.\; & S(\text{Alice}) \rightarrow P(\text{Alice})
			                 &                                             & \text{From 1 by universal instantiation}                  \\
			            4.\; & P(\text{Alice})
			                 &                                             & \text{From 2 and 3 by modus ponens}
		            \end{align*}

		            So the argument is \textbf{valid}.

		      \item \emph{``No juniors were on campus on the weekend'' and
			            ``Some ICT students are not juniors'' imply the conclusion
			            ``Some ICT students were on campus on the weekend''.}

		            \textbf{Predicates.}
		            Domain: all ICT students.
		            \[
			            J(x): x \text{ is a junior},\qquad
			            C(x): x \text{ was on campus on the weekend}.
		            \]

		            \textbf{Formal premises.}
		            \begin{align*}
			            1.\; & \forall x\,(J(x) \rightarrow \neg C(x))
			                 &                                         & \text{No juniors were on campus.}         \\
			            2.\; & \exists x\,(\neg J(x))
			                 &                                         & \text{Some ICT students are not juniors.}
		            \end{align*}

		            The desired conclusion is
		            \[
			            3.\; \exists x\, C(x).
		            \]

		            There is no rule of inference that lets us deduce $C(x)$ (being on campus)
		            merely from $\neg J(x)$ (not being a junior).  For instance, both premises
		            are true in a situation where \emph{no} students were on campus at the weekend.
		            In that case the conclusion is false.

		            Hence the argument is \textbf{not valid}.

		            \newpage
		      \item \emph{All parrots like fruit. My pet bird is not a parrot.
			            Therefore, my pet bird does not like fruit.}

		            \textbf{Predicates.}
		            Domain: all birds.
		            \[
			            P(x): x \text{ is a parrot},\qquad
			            F(x): x \text{ likes fruit}.
		            \]

		            \textbf{Formal premises.}
		            \begin{align*}
			            1.\; & \forall x\,(P(x) \rightarrow F(x)) &  & \text{All parrots like fruit}                 \\
			            2.\; & \neg P(b)                          &  & \text{$b$ is my pet bird and is not a parrot}
		            \end{align*}

		            Desired conclusion: $\neg F(b)$.

		            But from $P(x) \rightarrow F(x)$ and $\neg P(b)$ we cannot infer $\neg F(b)$;
		            this would be the fallacy of \emph{denying the antecedent}.  It is entirely
		            possible that $b$ likes fruit even though it is not a parrot.

		            Therefore this argument is \textbf{not valid}.

		      \item \emph{Everyone who eats granola every day is healthy.
			            Linda is not healthy. Therefore, Linda does not eat granola every day.}

		            \textbf{Predicates.}
		            Domain: all people.
		            \[
			            G(x): x \text{ eats granola every day},\qquad
			            H(x): x \text{ is healthy}.
		            \]

		            \textbf{Formal argument.}
		            \begin{align*}
			            1.\; & \forall x\,(G(x) \rightarrow H(x))
			                 &                                             & \text{Premise}                           \\
			            2.\; & \neg H(\text{Linda})
			                 &                                             & \text{Premise}                           \\
			            3.\; & G(\text{Linda}) \rightarrow H(\text{Linda})
			                 &                                             & \text{From 1 by universal instantiation} \\
			            4.\; & \neg G(\text{Linda})
			                 &                                             & \text{From 2 and 3 by modus tollens}
		            \end{align*}

		            Thus the argument is \textbf{valid}.

	      \end{enumerate}

	      %------------------------------------------------------------
	      \newpage
	\item Provided below is a formal proof that supposedly shows that if
	      $\exists x\,P(x) \land \exists x\,Q(x)$ is true, then $\exists x\,(P(x) \land Q(x))$ is true.
	      This implication is actually \emph{not} valid.

	      For each step in the argument, state whether the step is valid or not.  If it is not valid,
	      briefly explain why.

	      \medskip

	      \begin{tabular}{r l l}
		      1. & $\exists x\,P(x) \land \exists x\,Q(x)$ & Premise                             \\
		      2. & $\forall x\,P(x)$                       & Simplification from (1)             \\
		      3. & $P(c)$                                  & Universal instantiation from (2)    \\
		      4. & $Q(x)$                                  & Simplification from (1)             \\
		      5. & $Q(c)$                                  & Existential generalisation from (4) \\
		      6. & $P(c) \land Q(c)$                       & Simplification from (3) and (5)     \\
		      7. & $\exists x\,(P(x) \land Q(x))$          & Existential generalisation from (6)
	      \end{tabular}

	      \medskip

	      \textbf{Step-by-step analysis.}

	      \begin{itemize}
		      \item \textbf{Step 1.} This is just the premise, so it is \textbf{fine}.

		      \item \textbf{Step 2.} \emph{Invalid.}
		            From $\exists x\,P(x) \land \exists x\,Q(x)$ we may use simplification to obtain
		            $\exists x\,P(x)$ or $\exists x\,Q(x)$, but \emph{not} $\forall x\,P(x)$.
		            The quantifier has been incorrectly changed from existential to universal.

		      \item \textbf{Step 3.} \emph{Invalid as used here.}
		            The rule that fits $\exists x\,P(x)$ is \emph{existential instantiation},
		            not universal instantiation.  Properly, from $\exists x\,P(x)$ we may infer $P(c)$
		            for some fresh constant $c$, but only if line 2 were $\exists x\,P(x)$.
		            As written, it is using the wrong rule on an incorrectly obtained line.

		      \item \textbf{Step 4.} \emph{Invalid.}
		            Again, simplification from line 1 gives $\exists x\,Q(x)$, not $Q(x)$
		            with a free variable.  The expression $Q(x)$ would say that \emph{every}
		            element satisfies $Q$, which is much stronger than the premise.

		      \item \textbf{Step 5.} \emph{Invalid.}
		            Existential generalisation goes from a statement about a specific element,
		            such as $Q(c)$, to $\exists x\,Q(x)$, not the other way round.
		            Here the direction of the rule is reversed.

		      \item \textbf{Step 6.} \emph{Invalid.}
		            Even if we had correctly derived $P(c_{1})$ from $\exists x\,P(x)$
		            and $Q(c_{2})$ from $\exists x\,Q(x)$, the constants $c_{1}$ and $c_{2}$
		            need not be the same individual.  We cannot simply combine them into
		            $P(c) \land Q(c)$ with a single constant.

		      \item \textbf{Step 7.} \emph{Invalid as a consequence of earlier errors.}
		            From a correctly obtained $P(c) \land Q(c)$ it would be valid to infer
		            $\exists x\,(P(x) \land Q(x))$ by existential generalisation.  However,
		            since step 6 is not justified, this final step does not yield a valid
		            conclusion for the original premises.
	      \end{itemize}

	      \medskip
	      Overall, the argument fails because $\exists x\,P(x)$ and $\exists x\,Q(x)$
	      may be satisfied by \emph{different} elements; there need not be a single
	      element that makes both $P$ and $Q$ true.

\end{enumerate}

\end{document}
