\documentclass[12pt]{scrartcl}

\usepackage{amsmath,amssymb}
\usepackage{fullpage}
\usepackage{enumitem}
\usepackage{hyperref}

\setlength{\parindent}{0pt}

\begin{document}

\begin{center}
    \hrule
    \vspace{0.4cm}
    {\textbf{\large INF1003 Tutorial 10}}\\[0.2cm]
\end{center}

\textbf{Name:} Timothy Chia\hspace{\fill} \textbf{Topic:} Relations\\
\textbf{Student ID:} 2501530 \hspace{\fill} \textbf{Due Date:} 16/11/2025, 10:00 PM\\

\hrule

\begin{enumerate}[label=\textbf{\arabic*.}]

%------------------------------------------------------------
\item
Let $M_R$ be the matrix of a relation $R$ (with respect to the ordering
$1,2,3,4$ of the underlying set $A=\{1,2,3,4\}$):
\[
M_R =
\begin{pmatrix}
1 & 0 & 1 & 0 \\
1 & 1 & 0 & 1 \\
1 & 1 & 1 & 0 \\
1 & 1 & 0 & 1
\end{pmatrix}.
\]

\medskip

Determine whether $R$ is reflexive, symmetric, antisymmetric, or transitive.

\medskip

\textbf{Reflexive.}
$R$ is reflexive on $A$ iff all diagonal entries are $1$:
$M_{11}=M_{22}=M_{33}=M_{44}=1$.
From the matrix this holds, so $R$ is \emph{reflexive}.

\textbf{Symmetric.}
$R$ is symmetric iff $M_{ij} = M_{ji}$ for all $i,j$.
Here $M_{12}=0$ but $M_{21}=1$, so $M_R \ne M_R^{\mathsf T}$.
Thus $R$ is \emph{not symmetric}.

\textbf{Antisymmetric.}
$R$ is antisymmetric iff for all $i \ne j$,
if $M_{ij}=1$ and $M_{ji}=1$ then $i=j$ (contradiction).
But $M_{13}=1$ and $M_{31}=1$ with $1 \ne 3$.
Hence $R$ is \emph{not antisymmetric}.

\textbf{Transitive.}
$R$ is transitive iff whenever $(i,j)$ and $(j,k)$ are in $R$,
then $(i,k)$ is also in $R$.
From the matrix:
\[
(1,3) \in R\ (\text{since }M_{13}=1),\quad
(3,2) \in R\ (\text{since }M_{32}=1),
\]
but
\[
(1,2) \notin R\ (\text{since }M_{12}=0).
\]
So the transitivity condition fails. Therefore $R$ is \emph{not transitive}.

%------------------------------------------------------------
\newpage
\item
Let $R$ be the relation on $A = \{1,2,3,4\}$ defined by
\[
R = \{(2,2),(2,3),(2,4),(3,2),(3,3),(3,4)\}.
\]
Decide whether $R$ is reflexive, symmetric, antisymmetric, and/or transitive.

\medskip

\textbf{Reflexive?}
On $A$, reflexivity requires $(1,1),(2,2),(3,3),(4,4) \in R$.
Here only $(2,2)$ and $(3,3)$ are present; $(1,1)$ and $(4,4)$ are missing.
So $R$ is \emph{not reflexive}.

\textbf{Symmetric?}
Symmetry requires: whenever $(a,b)\in R$, then $(b,a)\in R$.
We have $(2,4)\in R$ but $(4,2)\notin R$,
so $R$ is \emph{not symmetric}.

\textbf{Antisymmetric?}
Antisymmetry requires: if $(a,b)\in R$ and $(b,a)\in R$ with $a\ne b$, this must never happen.
But $(2,3)\in R$ and $(3,2)\in R$ with $2\ne 3$, so $R$ is \emph{not antisymmetric}.

\textbf{Transitive?}
Transitivity: if $(a,b)\in R$ and $(b,c)\in R$, then $(a,c)\in R$.

We check all possible chains:

\begin{itemize}
    \item From $(2,2)$:
    \[
        (2,2),(2,2)\Rightarrow (2,2)\in R;\quad
        (2,2),(2,3)\Rightarrow (2,3)\in R;\quad
        (2,2),(2,4)\Rightarrow (2,4)\in R.
    \]
    \item From $(2,3)$:
    \[
        (2,3),(3,2)\Rightarrow (2,2)\in R;\quad
        (2,3),(3,3)\Rightarrow (2,3)\in R;\quad
        (2,3),(3,4)\Rightarrow (2,4)\in R.
    \]
    \item From $(3,2)$:
    \[
        (3,2),(2,2)\Rightarrow (3,2)\in R;\quad
        (3,2),(2,3)\Rightarrow (3,3)\in R;\quad
        (3,2),(2,4)\Rightarrow (3,4)\in R.
    \]
    \item From $(3,3)$:
    \[
        (3,3),(3,2)\Rightarrow (3,2)\in R;\quad
        (3,3),(3,3)\Rightarrow (3,3)\in R;\quad
        (3,3),(3,4)\Rightarrow (3,4)\in R.
    \]
    \item From $(2,4)$ or $(3,4)$ as first pair: there is no pair in $R$ whose first
          component is $4$, so no further chain to check.
\end{itemize}

In every case where $(a,b)$ and $(b,c)$ belong to $R$, $(a,c)$ is also in $R$.
Hence $R$ is \emph{transitive}.

%------------------------------------------------------------
\newpage
\item
For each of the relations in Question~3 (given as directed graphs on the worksheet),
determine whether the relation is reflexive, symmetric, antisymmetric, and/or transitive.
For any property that does not hold, give a brief explanation.

\medskip

Denote the three relations by $R_a$, $R_b$ and $R_c$.

\begin{enumerate}[label=(\alph*)]
    \item For $R_a$:
    \begin{itemize}
        \item \textbf{Reflexive:} Yes. Each element of the underlying set has a loop $(x,x)$,
              so every element is related to itself.
        \item \textbf{Symmetric:} Yes. For every arrow $x \to y$ in the digraph,
              there is also an arrow $y \to x$, so if $(x,y)\in R_a$ then $(y,x)\in R_a$.
        \item \textbf{Antisymmetric:} No. Since the relation is symmetric and there are
              distinct elements joined by two-way arrows, we have distinct $x\ne y$
              with both $(x,y)$ and $(y,x)$ in $R_a$, contradicting antisymmetry.
        \item \textbf{Transitive:} No. There is at least one path $x \to y \to z$
              where there is no direct edge $x \to z$, so $(x,y)$ and $(y,z)$ are in $R_a$
              but $(x,z)$ is not.
    \end{itemize}

    \item For $R_b$:
    \begin{itemize}
        \item \textbf{Reflexive:} No. At least one element has no loop $(x,x)$.
        \item \textbf{Symmetric:} No. There is at least one arrow $x \to y$ with
              no matching arrow $y \to x$.
        \item \textbf{Antisymmetric:} No. There exists a pair of distinct vertices
              $x\ne y$ with arrows in both directions, $(x,y)$ and $(y,x)$.
        \item \textbf{Transitive:} No. There is a two-step path $x \to y \to z$
              with no direct arrow $x \to z$.
    \end{itemize}

    \item For $R_c$:
    \begin{itemize}
        \item \textbf{Reflexive:} No. Some vertex misses its loop $(x,x)$.
        \item \textbf{Symmetric:} No. There is an arrow $x \to y$ without a reverse
              arrow $y \to x$.
        \item \textbf{Antisymmetric:} No. There is at least one pair of distinct
              vertices $x\ne y$ with arrows both ways.
        \item \textbf{Transitive:} No. Again there is a path $x \to y \to z$ with
              no edge $x \to z$.
    \end{itemize}
\end{enumerate}

%------------------------------------------------------------
\newpage
\item
Which of the following relations on the set of SIT students are equivalence relations?
For those that are, describe the equivalence classes. For each relation, briefly justify
your answer.

\begin{enumerate}[label=(\alph*)]
    \item $\{(a,b)\mid a \text{ and } b \text{ are taking a module together}\}$.

    Two students are related if they share at least one common module.

    \textbf{Reflexive:} If every student is enrolled in at least one module, then each
    student trivially shares a module with themselves, so $a$ is related to $a$.
    \textbf{Symmetric:} If $a$ and $b$ share a module, then $b$ and $a$ do as well.
    \textbf{Transitive:} Not necessarily. It can happen that $a$ and $b$ share module $X$,
    and $b$ and $c$ share module $Y\ne X$, but $a$ and $c$ do not share any module.
    Hence the relation is \emph{not transitive} and so \emph{not an equivalence relation}.

    \item $\{(a,b)\mid a \text{ and } b \text{ study in the same programme}\}$.

    \textbf{Reflexive:} Every student is in the same programme as themselves.  
    \textbf{Symmetric:} If $a$ is in the same programme as $b$, then $b$ is in the
    same programme as $a$.  
    \textbf{Transitive:} If $a$ and $b$ are in the same programme and $b$ and $c$
    are in the same programme, then $a$ and $c$ are in that same programme.

    Thus this is an \emph{equivalence relation}.  The equivalence classes are the sets
    of students in each programme (e.g. all AAI students, all SFT students, etc.).

    \item $\{(a,b)\mid \text{GPA of }a \text{ is greater than or equal to GPA of }b\}$.

    \textbf{Reflexive:} $\mathrm{GPA}(a)\ge \mathrm{GPA}(a)$, so $(a,a)$ is always in the relation.  
    \textbf{Symmetric:} Fails in general: if $\mathrm{GPA}(a)>\mathrm{GPA}(b)$ then
    $(a,b)$ is in the relation but $(b,a)$ is not.  
    \textbf{Transitive:} If $\mathrm{GPA}(a)\ge\mathrm{GPA}(b)$ and
    $\mathrm{GPA}(b)\ge\mathrm{GPA}(c)$, then $\mathrm{GPA}(a)\ge\mathrm{GPA}(c)$.

    Since it is not symmetric, this is \emph{not} an equivalence relation.

    \item $\{(a,b)\mid -0.5 \le \mathrm{GPA}(a)-\mathrm{GPA}(b) \le 0.5\}$.

    \textbf{Reflexive:} $\mathrm{GPA}(a)-\mathrm{GPA}(a)=0$, which lies in $[-0.5,0.5]$, so reflexive.  

    \textbf{Symmetric:} If
    $-0.5 \le \mathrm{GPA}(a)-\mathrm{GPA}(b) \le 0.5$, then
    $\mathrm{GPA}(b)-\mathrm{GPA}(a)=-(\mathrm{GPA}(a)-\mathrm{GPA}(b))$
    also lies in $[-0.5,0.5]$. So the relation is symmetric.

    \textbf{Transitive:} Not always.  
    Example: let
    \[
        \mathrm{GPA}(a)=3.0,\quad
        \mathrm{GPA}(b)=3.4,\quad
        \mathrm{GPA}(c)=3.8.
    \]
    Then $\mathrm{GPA}(a)-\mathrm{GPA}(b)=-0.4$ and
    $\mathrm{GPA}(b)-\mathrm{GPA}(c)=-0.4$ are both in $[-0.5,0.5]$, so
    $(a,b)$ and $(b,c)$ are related. But
    $\mathrm{GPA}(a)-\mathrm{GPA}(c)=-0.8$, which lies outside $[-0.5,0.5]$,
    so $(a,c)$ is not related. Hence the relation is \emph{not transitive}
    and so not an equivalence relation.

    \item $\{(a,b)\mid a \text{ is taking the same number of academic credits as } b\}$.

    \textbf{Reflexive:} Each student takes the same number of credits as themselves.  
    \textbf{Symmetric:} If $a$ takes the same number of credits as $b$, then $b$ takes
    the same number of credits as $a$.  
    \textbf{Transitive:} If $a$ and $b$ take the same number of credits and $b$ and $c$ also
    take the same number, then $a$ and $c$ take that same number.

    Thus this is an \emph{equivalence relation}.  The equivalence classes are the sets of
    students grouped by credit load (e.g. all students taking $20$ credits, all taking $15$ credits, etc.).
\end{enumerate}

%------------------------------------------------------------
\item
Let $R$ be the relation on the set of all strings of English letters such that
$aRb$ iff $l(a)=l(b)$, where $l(x)$ is the length of the string $x$.

\begin{enumerate}[label=(\alph*)]
    \item Is $R$ reflexive?  

    Yes.  For any string $a$, we have $l(a)=l(a)$, so $(a,a)\in R$.

    \item Is $R$ symmetric?  

    Yes.  If $l(a)=l(b)$, then $l(b)=l(a)$, so $(a,b)\in R$ implies $(b,a)\in R$.

    \item Is $R$ transitive?  

    Yes.  If $l(a)=l(b)$ and $l(b)=l(c)$, then $l(a)=l(c)$, so $(a,c)\in R$.

    \item Is $R$ an equivalence relation? If so, what are the equivalence classes?

    Since $R$ is reflexive, symmetric, and transitive, it is an \emph{equivalence relation}.
    The equivalence classes are sets of strings of the same length.  For each integer
    $n \ge 0$, there is one equivalence class consisting of all strings of length $n$.
\end{enumerate}

%------------------------------------------------------------
\item
Show that the ``divides'' relation on the set of positive integers is not an equivalence relation.

\medskip

Let the relation be $aRb$ iff $a$ divides $b$ (written $a \mid b$).

\textbf{Reflexive:} For every positive integer $a$, we have $a \mid a$, since $a = a \cdot 1$.
So the relation is reflexive.

\textbf{Transitive:} If $a \mid b$ and $b \mid c$, then there exist integers $m,n$ such that
$b = am$ and $c = bn = (am)n = a(mn)$, so $a \mid c$. Thus the relation is transitive.

\textbf{Symmetric:} Not in general.  For example $2 \mid 4$ (since $4 = 2 \cdot 2$),
but $4 \nmid 2$.  Therefore the relation is \emph{not symmetric}.

Since it fails symmetry, the divides relation is \emph{not} an equivalence relation.

%------------------------------------------------------------
\newpage
\item
Define a relation $R$ from $\mathbb{Z}$ to $\mathbb{Z}$ by
\[
(m,n) \in R \iff (m-n) \text{ is odd}.
\]

\begin{enumerate}[label=(\alph*)]
    \item Which of the following ordered pairs are in $R$?

    \begin{enumerate}[label=(\roman*)]
        \item $(8,2)$: $8-2 = 6$, which is even, so $(8,2) \notin R$.
        \item $(1,4)$: $1-4 = -3$, which is odd, so $(1,4) \in R$.
        \item $(5,-3)$: $5 - (-3) = 8$, which is even, so $(5,-3) \notin R$.
        \item $(3,2)$: $3-2 = 1$, which is odd, so $(3,2) \in R$.
    \end{enumerate}

    \item List five integers that are related by $R$ to $5$.

    We need integers $k$ such that $(k,5)\in R$, i.e. $k-5$ is odd.
    Because $5$ is odd, $k-5$ is odd exactly when $k$ is even.
    So all even integers are related to $5$.

    Examples of five such integers:
    \[
        0,\ 2,\ 4,\ 6,\ 8.
    \]

    \item Prove that if $n$ is any even integer, then $(n,3) \in R$.

    \textbf{Proof.}
    Let $n$ be an even integer.  Then there exists an integer $k$ such that $n = 2k$.
    Consider $n - 3$:
    \[
        n - 3 = 2k - 3 = 2k - 4 + 1 = 2(k-2) + 1.
    \]
    Since $k-2$ is an integer, $n-3$ is of the form $2(\text{integer}) + 1$, so it is odd.
    Therefore $(n,3)\in R$ by definition of $R$.
    \hfill$\Box$

    \item Is $R$ an equivalence relation? If so, how many equivalence classes does it have?

    We examine the usual properties on $\mathbb{Z}$.

    \textbf{Reflexive?}
    For reflexivity we would need $(n,n)\in R$ for all integers $n$.
    But $n-n=0$, which is even, so $(n,n)\notin R$ for every $n$.
    Thus $R$ is \emph{not reflexive}.

    \textbf{Symmetric?}
    If $(m,n)\in R$ then $m-n$ is odd.  Then $n-m = -(m-n)$ is also odd,
    so $(n,m)\in R$.  Hence $R$ is symmetric.

    \textbf{Transitive?}
    Not always.  Take $1,2,3$:
    \[
        1 - 2 = -1 \text{ (odd)} \Rightarrow (1,2)\in R,\quad
        2 - 3 = -1 \text{ (odd)} \Rightarrow (2,3)\in R,
    \]
    but
    \[
        1 - 3 = -2 \text{ (even)} \Rightarrow (1,3)\notin R.
    \]
    So $R$ is \emph{not transitive}.

    Since $R$ is not reflexive and not transitive, it is \emph{not} an equivalence relation.
    (So the question of equivalence classes does not apply.)
\end{enumerate}

\end{enumerate}

\end{document}
